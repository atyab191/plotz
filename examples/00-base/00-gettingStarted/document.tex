\documentclass{article}
\usepackage[a4paper, margin=2cm]{geometry}
\usepackage{plotz}

\title{Basic \texttt{PlotZ} example}
\date{}
\author{}

\begin{document}
\maketitle

Figure~\ref{fig:plotz} shows a typical \texttt{PlotZ} figure.

%plotz
\begin{figure}[h]
  \centering
  \plotz{plot}
  \caption{A \texttt{PlotZ} plot}
  \label{fig:plotz}
\end{figure}
%plotz

\vfill

In figure~\ref{fig:plotz*}, the same plot has been rescaled so that two
subfigures fit nicely side by side in the page. Notice how the font size, line
thickness, \textit{etc.} remain constant; only the axes scales are adjusted.

%plotz*
\begin{figure}[h]
  \centering
  \plotz[width=0.47\textwidth]{plot}%
  \hfill%
  \plotz[width=0.47\textwidth]{plot}%
  \caption{Two \texttt{PlotZ} pictures scaled down to fit side by side.}
  \label{fig:plotz*}
\end{figure}
%plotz*

\end{document}
